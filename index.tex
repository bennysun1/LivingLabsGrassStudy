% Options for packages loaded elsewhere
\PassOptionsToPackage{unicode}{hyperref}
\PassOptionsToPackage{hyphens}{url}
\PassOptionsToPackage{dvipsnames,svgnames,x11names}{xcolor}
%
\documentclass[
  letterpaper,
  DIV=11,
  numbers=noendperiod]{scrreprt}

\usepackage{amsmath,amssymb}
\usepackage{iftex}
\ifPDFTeX
  \usepackage[T1]{fontenc}
  \usepackage[utf8]{inputenc}
  \usepackage{textcomp} % provide euro and other symbols
\else % if luatex or xetex
  \usepackage{unicode-math}
  \defaultfontfeatures{Scale=MatchLowercase}
  \defaultfontfeatures[\rmfamily]{Ligatures=TeX,Scale=1}
\fi
\usepackage{lmodern}
\ifPDFTeX\else  
    % xetex/luatex font selection
\fi
% Use upquote if available, for straight quotes in verbatim environments
\IfFileExists{upquote.sty}{\usepackage{upquote}}{}
\IfFileExists{microtype.sty}{% use microtype if available
  \usepackage[]{microtype}
  \UseMicrotypeSet[protrusion]{basicmath} % disable protrusion for tt fonts
}{}
\usepackage{xcolor}
\setlength{\emergencystretch}{3em} % prevent overfull lines
\setcounter{secnumdepth}{5}
% Make \paragraph and \subparagraph free-standing
\ifx\paragraph\undefined\else
  \let\oldparagraph\paragraph
  \renewcommand{\paragraph}[1]{\oldparagraph{#1}\mbox{}}
\fi
\ifx\subparagraph\undefined\else
  \let\oldsubparagraph\subparagraph
  \renewcommand{\subparagraph}[1]{\oldsubparagraph{#1}\mbox{}}
\fi


\providecommand{\tightlist}{%
  \setlength{\itemsep}{0pt}\setlength{\parskip}{0pt}}\usepackage{longtable,booktabs,array}
\usepackage{calc} % for calculating minipage widths
% Correct order of tables after \paragraph or \subparagraph
\usepackage{etoolbox}
\makeatletter
\patchcmd\longtable{\par}{\if@noskipsec\mbox{}\fi\par}{}{}
\makeatother
% Allow footnotes in longtable head/foot
\IfFileExists{footnotehyper.sty}{\usepackage{footnotehyper}}{\usepackage{footnote}}
\makesavenoteenv{longtable}
\usepackage{graphicx}
\makeatletter
\def\maxwidth{\ifdim\Gin@nat@width>\linewidth\linewidth\else\Gin@nat@width\fi}
\def\maxheight{\ifdim\Gin@nat@height>\textheight\textheight\else\Gin@nat@height\fi}
\makeatother
% Scale images if necessary, so that they will not overflow the page
% margins by default, and it is still possible to overwrite the defaults
% using explicit options in \includegraphics[width, height, ...]{}
\setkeys{Gin}{width=\maxwidth,height=\maxheight,keepaspectratio}
% Set default figure placement to htbp
\makeatletter
\def\fps@figure{htbp}
\makeatother

\KOMAoption{captions}{tableheading}
\makeatletter
\makeatother
\makeatletter
\@ifpackageloaded{bookmark}{}{\usepackage{bookmark}}
\makeatother
\makeatletter
\@ifpackageloaded{caption}{}{\usepackage{caption}}
\AtBeginDocument{%
\ifdefined\contentsname
  \renewcommand*\contentsname{Table of contents}
\else
  \newcommand\contentsname{Table of contents}
\fi
\ifdefined\listfigurename
  \renewcommand*\listfigurename{List of Figures}
\else
  \newcommand\listfigurename{List of Figures}
\fi
\ifdefined\listtablename
  \renewcommand*\listtablename{List of Tables}
\else
  \newcommand\listtablename{List of Tables}
\fi
\ifdefined\figurename
  \renewcommand*\figurename{Figure}
\else
  \newcommand\figurename{Figure}
\fi
\ifdefined\tablename
  \renewcommand*\tablename{Table}
\else
  \newcommand\tablename{Table}
\fi
}
\@ifpackageloaded{float}{}{\usepackage{float}}
\floatstyle{ruled}
\@ifundefined{c@chapter}{\newfloat{codelisting}{h}{lop}}{\newfloat{codelisting}{h}{lop}[chapter]}
\floatname{codelisting}{Listing}
\newcommand*\listoflistings{\listof{codelisting}{List of Listings}}
\makeatother
\makeatletter
\@ifpackageloaded{caption}{}{\usepackage{caption}}
\@ifpackageloaded{subcaption}{}{\usepackage{subcaption}}
\makeatother
\makeatletter
\@ifpackageloaded{tcolorbox}{}{\usepackage[skins,breakable]{tcolorbox}}
\makeatother
\makeatletter
\@ifundefined{shadecolor}{\definecolor{shadecolor}{rgb}{.97, .97, .97}}
\makeatother
\makeatletter
\makeatother
\makeatletter
\makeatother
\ifLuaTeX
  \usepackage{selnolig}  % disable illegal ligatures
\fi
\IfFileExists{bookmark.sty}{\usepackage{bookmark}}{\usepackage{hyperref}}
\IfFileExists{xurl.sty}{\usepackage{xurl}}{} % add URL line breaks if available
\urlstyle{same} % disable monospaced font for URLs
\hypersetup{
  pdftitle={Living Labs Grass Study},
  pdfauthor={Ben Sunshine},
  colorlinks=true,
  linkcolor={blue},
  filecolor={Maroon},
  citecolor={Blue},
  urlcolor={Blue},
  pdfcreator={LaTeX via pandoc}}

\title{Living Labs Grass Study}
\author{Ben Sunshine}
\date{2024-01-25}

\begin{document}
\maketitle
\ifdefined\Shaded\renewenvironment{Shaded}{\begin{tcolorbox}[frame hidden, sharp corners, interior hidden, breakable, boxrule=0pt, enhanced, borderline west={3pt}{0pt}{shadecolor}]}{\end{tcolorbox}}\fi

\renewcommand*\contentsname{Table of contents}
{
\hypersetup{linkcolor=}
\setcounter{tocdepth}{2}
\tableofcontents
}
\bookmarksetup{startatroot}

\hypertarget{abstract}{%
\chapter{Abstract}\label{abstract}}

~~~~~~In Alaska, indigenous hunters and gatherers have long observed the
alignment of grass and plants after the growing season as indicative of
prevailing wind directions and shifts. Due to the remote and harsh
conditions, traditional weather stations are absent to measure shifts in
historically predominant wind directions. In a previous study Dr.~Jon
Rosales (Environmental Studies) and his team collected images of grass
lay from St.~Lawrence Island, Alaska, and manually attempted to measure
grass lay angles. This project investigated the Histogram of Oriented
Gradients (HOG) algorithm to automate this process. We applied the
algorithm to various images of grass fields sampled from the internet to
test its viability.

\bookmarksetup{startatroot}

\hypertarget{introduction}{%
\chapter{Introduction}\label{introduction}}

~~~~~~Subsistence-oriented indigenous communities across Alaska rely
heavily on Traditional Ecological Knowledge (TEK), a holistic
understanding of their environment acquired through generations of
observation and cultural transmission. Among the Anishinaabek tradition,
sweetgrass symbolizes wisdom and knowledge, passed down from elders to
younger generations. Indigenous hunters and gatherers have long observed
the alignment of grass and plants after the growing season as indicative
of prevailing wind directions. Predominant wind direction serve a
crucial role to subsistence practitioners when hunting, fishing,
settling, and keeping track of changing weather. On islands like
St.~Lawrence Island in Savoonga, AK, natives have observed a shift from
historically predominant northerly wind patterns to southerly and
easterly and dominated winds. This hypothesis has not yet been
confirmed, due to the absence of weather stations.\\
\hspace*{0.333em}\hspace*{0.333em}\hspace*{0.333em}\hspace*{0.333em}\hspace*{0.333em}\hspace*{0.333em}This
research project seeks to reinforce Traditional Ecological Knowledge
(TEK) with Scientific Ecological Knowledge (SEK) to develop our
understanding of Alaskan indigenous wisdom and its correlation with
modern scientific findings. By employing advanced image processing and
machine learning techniques on imaging data obtained from the Living
Laboratory at St.~Lawrence University, we aim to utilize a Histogram of
Oriented Gradients (HOG) algorithm to enhance the methodology of data
collection and measure the predominant direction in which grass lays
after its growing season. This interdisciplinary approach not only
deepens our comprehension of indigenous knowledge systems but also sheds
light on environmental dynamics and climate change impacts in remote
regions.

\bookmarksetup{startatroot}

\hypertarget{methods}{%
\chapter{Methods}\label{methods}}

~~~~~~The HOG algorithm, introduced by Navneet Dalal and Bill Triggs in
2005, is a popular technique for object detection in images. The
algorithm can identify gradient magnitudes and angles at each pixel in
an image. The preliminary steps involved using the `skimage' library
from Python to preprocess the images of interest. This included loading,
resizing, and converting the images to grayscale. Images were rescaled
to standardize their resolutions and preserve their aspect ratios to
prevent distortion that could affect the accuracy of angle
identification. Converting the images to grayscale was necessary because
it allowed for focusing on a single channel to represent pixel
intensity, rather than three channels (red, green, and blue).\\
\hspace*{0.333em}\hspace*{0.333em}\hspace*{0.333em}\hspace*{0.333em}\hspace*{0.333em}\hspace*{0.333em}The
HOG features were then computed for the resized images, which involved
calculating the gradient magnitudes and angles at each pixel. The
gradient magnitude at each pixel is comprised of the gradients in the
`x' and `y' directions. The gradient in the x-direction is computed by
subtracting the pixel value to the left of pixel of interest is
subtracted from the pixel value to its right. Similarly, the gradient in
the y-direction is calculated by pixel value below the pixel of interest
is subtracted from the pixel value above the pixel of interest. Now to
calculate the gradient magnitude at the pixel of interest, the
Pythagorean Theorem can be utilized where the gradient magnitude is
equal to the square root of the x-gradient squared plus the y-gradient
squared. The angle at a given pixel can be calculated by taking the
inverse tangent of its y-gradient divided by its x-gradient. It is
important to note all angles produced by this algorithm are between zero
and one hundred eighty degrees. This occurs, because the inverse tangent
function used for calculating a given pixel's angle cannot distinguish
between all four quadrants. OR:

\(Magnitude(\mu)=\sqrt{G_{x}^{2} + G_{y}^{2}}\)

\(Angle(\Theta)=\arctan({\frac{G_{y}^{2}}{G_{x}^{2}}})\)

~~~~~~Next, histograms are constructed to visualize the distribution of
gradient magnitudes and angles. Two different techniques for creating
gradient angle histograms were implemented. The second scheme uses the
same number of bins, but factors in a pixel's gradient magnitude and its
allocation to its bordering bins. Here, the weight assigned to each bin
is calculated by the angle's deviation from the center of its central
bin. This approach allows for a more representative histogram which
splits angles between bins and takes their magnitudes into account.

~~~~~~Lastly, these histograms are converted to polar histograms so the
primary angles can be visualized and compared to their original images.

\bookmarksetup{startatroot}

\hypertarget{results}{%
\chapter{Results}\label{results}}

\bookmarksetup{startatroot}

\hypertarget{conclusion}{%
\chapter{Conclusion}\label{conclusion}}



\end{document}
